\documentclass[a4paper,12pt]{article}
\pagenumbering{gobble}
\usepackage{sectsty}
\usepackage{enumitem}
\usepackage[tmargin=1in,bmargin=0.5in]{geometry}
\usepackage[T1,T2A]{fontenc}
\usepackage[utf8x]{inputenc}
\usepackage[pdftex,unicode=true]{hyperref}
\usepackage{graphicx}
\usepackage{tikz}

\begin{document}

% \begin{tikzpicture}[remember picture,overlay]
%   \node [xshift=-4.9cm,yshift=-4.7cm] at (current page.north east)
%   {\includegraphics{photo.png}};
% \end{tikzpicture}

\section*{Mark Karpov}

Tel.: (+7) 960-947-8031\\
Email: markkarpov92@gmail.com\\
GitHub: \href{https://github.com/mrkkrp}{https://github.com/mrkkrp}\\
Twitter: \href{https://twitter.com/mrkkrp}{https://twitter.com/mrkkrp}\\
LinkedIn: \href{https://www.linkedin.com/in/mark-karpov-73789914a}{https://www.linkedin.com/in/mark-karpov-73789914a}\\
StackOverflow: \href{https://stackoverflow.com/users/3687048/mark}{https://stackoverflow.com/users/3687048/mark}\\
The resume is also available online: \href{https://markkarpov.com/resume.html}{https://markkarpov.com/resume.html}

\sectionfont{\fontsize{12}{15}\selectfont\sectionrule{0pt}{0pt}{-5pt}{0.8pt}}

\section*{Summary}

Software engineer specializing on functional programming using strong static
typing, in Haskell in particular.

\section*{Tech work experience}

\begin{itemize}[noitemsep]
\item \textbf{May 4, 2016--May 20, 2017} Tier 2 Haskell developer at
  \href{https://www.stackbuilders.com/}{Stack Builders} (remote). The job
  involved working for several clients mostly from the US. Web-applications
  built with Yesod, Snap, Servant (including a micro-service based system
  with Servant on backend). Agile. Another part of the job was writing
  tutorials and blog posts to increase visibility of the company.
\item \textbf{January 25, 2018--present} Software engineer at
  \href{https://tweag.io}{Tweag I/O} (remote).
\end{itemize}

\section*{Haskell}

\begin{itemize}[noitemsep]
\item Concepts (not mentioning basic things like rank-N types, existentials,
  phantom types etc.): EDSL using combinators, GADTs, type-level
  programming, generics, TH, high-performance Haskell, parallel and
  concurrent Haskell, FFI, lens (van Laarhoven/profunctor).
\item Libraries (not mentioning vital common libs like monad-control and
  base): lens, aeson, conduit, postgresql-simple, persistent, esqueleto,
  dbmigrations, parsec, megaparsec, attoparsec, yesod, snap, servant,
  http-client, http-conduit, wreq, req, cryptonite, warp, websockets,
  socket, network, HUnit, hspec, QuickCheck, test-framework, tasty,
  hedgehog, webdriver, optparse-applicative, path, path-io, stache, vector,
  containers, unordered-containers, binary, cereal, store,
  distributed-process, reflection, constraints, kan-extensions, etc. I have
  experience with most of the libraries in active use in the Haskell
  ecosystem.
\end{itemize}

\pagebreak

\section*{Technologies}

\begin{itemize}[noitemsep]
\item Databases: PostgreSQL, MySQL.
\item AWS.
\item Linux server.
\item Strong knowledge of Git.
\item CI: Travis CI, Circle CI.
\item Dependent types, theorem provers: Coq.
\item Bash/Python/Haskell scripting.
\item Build systems: make, shake, Bazel.
\item Server automation: Ansible.
\item Front-end: HTML (5), CSS (Bootstrap), JavaScript (jQuery, AJAX,
  websockets).
\item Other languages I know: C, C++, Python, Common Lisp, Emacs Lisp.
\end{itemize}

\section*{Open source projects}

\begin{itemize}[noitemsep]
\item \href{https://github.com/mrkkrp/megaparsec}
  {Megaparsec---Industrial-strength monadic parser combinator library.
    \newline
    https://github.com/mrkkrp/megaparsec}
\item \href{https://github.com/mrkkrp/req}
  {Req---Easy-to-use, type-safe, expandable, high-level HTTP library.
    \newline
    https://github.com/mrkkrp/req}
\item \href{https://github.com/mrkkrp/zip}
  {Zip---Efficient library for manipulating zip archives.
    \newline
    https://github.com/mrkkrp/zip}
\item \href{https://github.com/mmark-md/mmark}
  {MMark---Strict markdown processor for writers.
    \newline
    https://github.com/mmark-md/mmark}
\end{itemize}

The full list can be found at \href{https://markkarpov.com/oss.html}{https://markkarpov.com/oss.html}.

\section*{Writing}

\begin{itemize}[noitemsep]
\item I've authored a number of tutorials as part of my job and on my own,
  see e.g.
  \href{https://www.stackbuilders.com/tutorials/haskell/ghc-optimization-and-fusion/}{“GHC
    optimization and fusion”}, the full list is available at \\
  \href{https://markkarpov.com/learn-haskell.html}{https://markkarpov.com/learn-haskell.html}.
\item Blog posts are available at
  \href{https://markkarpov.com/posts.html}{https://markkarpov.com/posts.html},
  see e.g.
  \href{https://markkarpov.com/post/free-monad-considered-harmful.html}{“Free
    monad considered harmful”} from the recent ones.
\item I've authored two chapters for the upcoming book
  \href{https://intermediatehaskell.com/}{“Intermediate Haskell”}:
  Exceptions and Megaparsec. Unfortunately the content is not publicly
  available yet (only available to reviewers).
\end{itemize}

\section*{Education}

\begin{itemize}[noitemsep]
\item 2009--2014---Polzunov Altai State Technical University. Engineer
  degree in informational technology and measuring engineering.
\end{itemize}

\begin{flushright}
  \today
\end{flushright}

\end{document}
